%%% -*-LaTeX-*-
%%% wallisformula.tex.1
%%% Prettyprinted by texpretty lex version 0.02 [21-May-2001]
%%% on Thu Nov  2 05:32:09 2017
%%% for Steve Dunbar (sdunbar@family-desktop)

\documentclass[12pt]{article}

\input{../../../../etc/macros}
\input{../../../../etc/mzlatex_macros}
%% \input{../../../../etc/pdf_macros}

\bibliographystyle{plain}

\begin{document}

\myheader \mytitle

\hr

\sectiontitle{Wallis' Formula}

\hr

\usefirefox

\hr

\visual{Rating}{../../../../CommonInformation/Lessons/rating.png}
\section*{Rating} %one of
% Everyone: contains no mathematics.
%Student:  contains scenes of mild algebra or calculus that may require
%guidance.
Mathematically Mature:  may contain mathematics beyond calculus with
proofs. % Mathematicians Only: prolonged scenes of intense rigor.

\hr

\visual{Section Starter Question}{../../../../CommonInformation/Lessons/question_mark.png}
\section*{Section Starter Question} Can you think of a sequence or a
process that approximates \( \PI \)?  What is the intuition or reasoning
behind that sequence?

\hr

\visual{Key Concepts}{../../../../CommonInformation/Lessons/keyconcepts.png}
\section*{Key Concepts}

\begin{enumerate}
    \item
        \defn{Wallis' Formula} is the amazing limit
        \[
            \lim_{n \to \infty} \left( \frac{ 2 \cdot 2 \cdot 4 \cdot 4
            \cdot 6 \cdot 6 \dots (2n) \cdot (2n)}{1 \cdot 3 \cdot 3
            \cdot 5 \cdot 5 \dots (2n-1) \cdot (2n-1)\cdot(2n+1)} \right)
            = \frac{\PI}{2}.
        \]
    \item
        One proof of Wallis' formula uses a recursion formula developed
        from integration of trigonometric functions.
    \item
        Another proof uses only basic algebra, the Pythagorean Theorem,
        and the formula \( \PI r^2 \) for the area of a circle of radius
        \( r \).
    \item
        Yet another proof uses Euler's infinite product representation
        for the sine function.
\end{enumerate}

\hr

\visual{Vocabulary}{../../../../CommonInformation/Lessons/vocabulary.png}
\section*{Vocabulary}
\begin{enumerate}
    \item
        \defn{Wallis' Formula} is the amazing limit
        \[
            \lim_{n \to \infty} \left( \frac{ 2 \cdot 2 \cdot 4 \cdot 4
            \cdot 6 \cdot 6 \dots (2n) \cdot (2n)}{1 \cdot 3 \cdot 3
            \cdot 5 \cdot 5 \dots (2n-1) \cdot (2n-1)\cdot(2n+1)} \right)
            = \frac{\PI}{2}.
        \]
\end{enumerate}

\hr

\visual{Mathematical Ideas}{../../../../CommonInformation/Lessons/mathematicalideas.png}
\section*{Mathematical Ideas}
\subsection*{Introduction} \defn{Wallis' Formula} is the amazing limit
\begin{equation}
    \lim_{n \to \infty} \left( \frac{2 \cdot 2 \cdot 4 \cdot 4 \cdot 6
    \cdot 6 \dots (2n) \cdot (2n)}{1 \cdot 3 \cdot 3 \cdot 5 \cdot 5
    \dots (2n-1) \cdot (2n-1)\cdot(2n+1)} \right) = \frac{\PI}{2}.%
    \label{eq:wallisformula:wallis}
\end{equation}
%
\index{Wallis Formula}
Another way to write this is
\[
    \frac{\PI}{2} = \prod_{j=1}^{\infty} \frac{(2j)^2}{(2j-1)(2j+1)}.
\] A ``closed form'' expression for the product in Wallis formula is
\[
    \lim_{n \to \infty} \frac{ 2^{4n} \cdot (n!)^4}{((2n)!)^2(2n+1)} =
    \frac{\PI}{2}
\] or equivalently
\[
    \lim_{n \to \infty} \frac{ 2^{4n}}{{\binom{2n}{n}}^2 (2n+1)} = \frac
    {\PI}{2}.
\]

Note that Wallis Formula is equivalent to saying that the ``central
binomial term'' has the asymptotic expression
\[
    \frac{1}{2^{2n}} \binom{2n}{n} \asympt \sqrt{ \frac{2}{(2n+1)\PI} }.
\] See the subsection Central Binomial below for a proof of an
equivalent inequality.

In the form
\begin{equation}
    \label{eq:wallisformula:centralbinomexpansion} w_n = \prod_{j=1}^{n}
    \frac{(2j)^2}{(2j-1)(2j+1)} = \frac{2^{4n}}{%
    {\binom{2n}{n}}^2(2n+1)}
\end{equation}
it is easy to see that the sequence \( w_n \) is increasing since \( 4n^2/
(4n^2-1) > 1 \).  Figure~%
\ref{fig:wallisformula:wallis_plot} illustrates the increasing sequence.

\begin{figure}
    \centering
    \begin{asy}
settings.outformat = "pdf";

import graph;

size(5inches, IgnoreAspect);

real myfontsize = 10;
real mylineskip = 1.2*myfontsize;
pen mypen = fontsize(myfontsize, mylineskip);
defaultpen(mypen);

real halfpi = 2*atan(1);

real w(int n) {
real w1 = 1;
for (int j=1; j<n; ++j) {w1 = w1*(2*j)^2/((2*j-1)*(2*j+1)); }
return w1;
}

xaxis("$n$", xmin=0, xmax=20, RightTicks(20), arrow=Arrow());
yaxis("$w_n$", ymin=0, ymax=2, LeftTicks(4), arrow=Arrow());

yequals(Label("$\pi/2$", (0,halfpi), 3W), halfpi, xmin=0, xmax=20, blue);

for (int i=1; i<21; ++i)
dot((i,w(i)), red);
    \end{asy}
    \caption{Convergence of the Wallis formula to $ \PI/2 $.  %
    \label{fig:wallisformula:wallis_plot}}
\end{figure}

Doing a numerical linear regression of \( \log (\PI/2 - w_n) \) versus \(
\log n \) on the domain \( n = 1 \) to \( n = 30 \) indicates that \( w_n
\) approaches \( \PI/2 \) at a rate which is \( O(1/n) \).

\subsection*{A proof using integration and recursion}

\begin{theorem}[Wallis' Formula]
    \begin{equation}
        \lim_{n \to \infty} \left( \frac{2 \cdot 2 \cdot 4 \cdot 4 \cdot
        6 \cdot 6 \dots (2n) \cdot (2n)}{1 \cdot 3 \cdot 3 \cdot 5 \cdot
        5 \dots (2n-1)\cdot(2n+1)} \right) = \frac{\PI}{2}.
    \end{equation}
\end{theorem}
\index{Wallis Formula}

\begin{proof}
    Consider \( J_n = \int_0^{\PI/2} \cos^n(x) \, \df{x} \).
    Integrating by parts with \( u = \cos^{n-1}(x) \) and \( \df{v} =
    \cos(x) \) shows
    \begin{align*}
        \int_0^{\PI/2} \cos^n(x) \, \df{x} &= (n-1) \int_0^{\PI/2} \cos^
        {n-2}(x) \sin^2(x) \, \df{x} \\
        &= (n-1) \int_0^{\PI/2} \cos^{n-2}(x)(1 - \cos^2(x)) \, \df{x}
        \\
        &= (n-1) \int_0^{\PI/2} \cos^{n-2}(x) \, \df{x} - (n-1) \int_0^{\PI/2}
        \cos^n(x) \, \df{x}.
    \end{align*}
    Gathering terms, we get \( n J_n = (n-1) J_{n-2} \).

    Now \( J_1 = 1 \) so recursively \( J_3 = \frac{2}{3} \), \( J_5 =
    \frac{2 \cdot 4}{3 \cdot 5} \) and inductively
    \begin{equation}
        \label{eq:wallisformula:oddproduct} J_{2n+1} = \frac{2 \cdot 4
        \cdots (2n-2) \cdot (2n) }{1 \cdot 3 \cdots (2n-1) \cdot (2n+1)}.
    \end{equation}
    Likewise \( J_2 = \frac{\PI}{2\cdot2} \), and \( J_4 = \frac{3 \cdot
    \PI}{ 2 \cdot 4 \cdot 2} \), \( J_6 = \frac{3 \cdot 5 \cdot \PI}{2
    \cdot 4 \cdot 6 \cdot 2} \) and inductively
    \[
        J_{2n} = \frac{3 \cdot 5 \cdots (2n-3) \cdot (2n-1) \cdot \PI}{2
        \cdot 4 \cdots (2n-2) \cdot (2n) \cdot 2}.
    \]

    For \( 0 \le x \le \PI/2 \), \( 0 \le \cos(x) \le 1 \), so \( \cos^{2n}
    (x) \ge \cos^{2n+1}(x) \ge \cos^{2n+2}(x) \), implying in turn that \(
    J_{2n} \ge J_{2n+1} \ge J_{2n+2} \).  Then
    \[
        1 \ge \frac{J_{2n+1}}{J_{2n}} \ge \frac{J_{2n+2}}{J_{2n}} =
        \frac{2n+1}{2n+2}.
    \] Hence \( \lim_{n \to \infty} \frac{J_{2n+1}}{J_{2n}} = 1 \).

    That is,
    \[
        \lim_{n \to \infty} \left( \frac{ 2 \cdot 2 \cdot 4 \cdot 4
        \cdot 6 \cdot 6 \dots (2n) \cdot (2n)}{1 \cdot 3 \cdot 3 \cdot 5
        \cdot 5 \dots (2n-1)\cdot(2n+1)} \frac{2}{\PI} \right) = 1
    \] or equivalently
    \[
        \lim_{n \to \infty} \left( \frac{ 2 \cdot 2 \cdot 4 \cdot 4
        \cdot 6 \cdot 6 \dots (2n) \cdot (2n)}{1 \cdot 3 \cdot 3 \cdot 5
        \cdot 5 \dots (2n-1)\cdot(2n+1)} \right) = \frac{\PI}{2}.
    \]
\end{proof}

\subsection*{An elementary proof of Wallis formula}

The following proof is adapted and expanded from~%
\cite{wastlund07}.  The proof uses only basic algebra, the Pythagorean
Theorem, and the formula \( \PI r^2 \) for the area of a circle of
radius \( r \).  Another important property used implicitly is the
completeness property of the reals.

Define a sequence of numbers by \( s_1 = 1 \) and for \( n \ge 2 \),
\begin{equation}
    \label{eq:wallisformula:defnsn} s_n = \frac{3}{2} \cdot \frac{5}{4}
    \cdots \frac{2n-1}{2n-2}.
\end{equation}
These are the reciprocals of the subsequence \( J_{2n-1} \) defined in
equation \eqref{eq:wallisformula:oddproduct}.

The partial products of Wallis' formula \eqref{eq:wallisformula:wallis}
with an odd number of terms in the numerator are
\begin{equation}
    \label{eq:wallisformula:defnodds} o_n = \frac{2^2 \cdot 4^2 \cdots (2n-2)^2
    \cdot 2n}{1 \cdot 3^2 \cdots (2n-1)^2} = \frac{2n}{s_n^2},
\end{equation}
while those with an even number of factors in the numerator are of the
form
\begin{equation}
    \label{eq:wallisformula:defnevens} e_n = \frac{2^2 \cdot 4^2 \cdots
    (2n-2)^2}{1 \cdot 3^2 \cdots (2n-3)^2 \cdot (2n-1)} = \frac{2n-1}{s_n^2}.
\end{equation}
Interpret \( e_1 = 1 \) as an empty product.  Since \( \frac{(2n)^2}{(2n-1)
(2n+1)} > 1 \), clearly \( e_n < e_{n+1} \).  Since \( \frac{(2n)(2n+2)}
{(2n+1)^2} < 1 \), clearly \( o_n > o_{n+1} \).  Also, \( e_n < o_n \).
Therefore
\[
    e_1 < e_2 < e_3 < \cdots e_n < o_n < \cdots < o_3 < o_2 < o_1
\] for any \( n \).  Furthermore, for \( 1 \le i \le n \),
\[
    \frac{2i}{s_i^2} = o_i \ge o_n
\] and
\[
    \frac{2i-1}{s_i^2} = e_i \le e_n
\] from which it follows that
\begin{equation}
    \label{eq:wallisformula:oddeveninequality} \frac{2i-1}{e_n} \le s_i^2
    \le \frac{2i}{o_n}.
\end{equation}
For convenience, define \( s_0 = 0 \) so that inequality \eqref{eq:wallisformula:oddeveninequality}
holds also for \( i = 0 \). Denote the successive difference \( a_n = s_
{n+1} - s_n \) so \( a_0 = s_1 - s_0 = 1 \) and for \( n \ge 1 \)
\[
    a_n = s_{n+1} - s_n = s_n\left( \frac{2n+1}{2n} - 1 \right) = \frac{s_n}
    {2n} = \frac{1}{2} \cdot \frac{3}{4} \cdots \frac{2n-1}{2n}.
\]

\begin{lemma}
    For the sequence \( a_i \) we have the identity
    \begin{equation}
        \label{eq:wallisformula:identity} a_i a_j = \frac{j+1}{i+j+1} a_i
        a_{j+1} + \frac{i+1}{i+j+1} a_{i+1} a_{j}
    \end{equation}
    for any \( i \) and \( j \).
\end{lemma}

\begin{proof}
    Make the substitutions
    \[
        a_{i+1} = \frac{2i +1}{2(i+1)} a_i
    \] and
    \[
        a_{j+1} = \frac{2j +1}{2(j+1)} a_j
    \] the right side of the proposed identity \eqref{eq:wallisformula:identity}
    becomes
    \[
        a_i a_j \left( \frac{2j+1}{2(j+1)} \cdot \frac{j+1}{i+j+1} +
        \frac{2i+1}{2(i+1)} \cdot \frac{i+1}{i+j+1}\right) = a_i a_j.
    \]
\end{proof}

\begin{lemma}
    \begin{align*}
        1 = a_0^2 &= a_0 a_1 + a_1 a_0 \\
        &= a_0 a_2 + a_1^2 + a_2 a_0 \\
        & \ldots \\
        &= a_0 a_n + a_1 a_{n-1} + \cdots + a_n a_0
    \end{align*}
\end{lemma}

\begin{proof}
    Start from \( a_0^2 = 1 \) and repeatedly apply the identity \eqref{eq:wallisformula:identity}.
    At stage \( n \) applying the identity \eqref{eq:wallisformula:identity}
    to every term the sum
    \[
        a_0 a_n + a_1 a_{n-1} + \cdots + a_n a_0
    \] becomes
    \[
        \left( a_0 a_n + \frac{1}{n}a_1 a_{n-1}\right) + \left( \frac{n-1}
        {n} a_1 a_{n-1} + \frac{2}{n} a_2 a_{n-2}\right) + \cdots +
        \left( \frac{1}{n} a_{n-1} a_{1} + a_n a_{0} \right).
    \] Collecting terms, this simplifies to \( a_0 a_n + a_1 a_{n-1} +
    \cdots + a_n a_0 \).
\end{proof}

Now divide the positive quadrant of the \( xy \)-plane into rectangles
by drawing the vertical lines \( y = s_n \) and the horizontal lines \(
y = s_n \) for all \( n \).  Let \( R_{i,j} \) be the rectangle with
lower left corner \( ( s_i, s_j) \) and upper right corner \( (s_{i+1},
s_{j+1}) \).  The area of \( R_{i,j} \) is \( a_i a_j \).  Thus the
identity \( 1 = a_0 a_n + a_1 a_{n_1} + \cdots + a_n a_0 \) states that
the total area of the rectangles \( R_{i,j} \) for which \( i + j = n \)
is \( 1 \).  Let \( P_n \) be the polygonal region consisting of all
rectangles \( R_{i,j} \) for which \( i +j < n \).  Hence the area of \(
P_n \) is \( n \).

The outer corners of \( P_n \) are the points \( (s_i, s_j) \) for which
\( i + j = n+1 \) and \( 1 \le i,j, \le n \).  By the Pythagorean
theorem, the distance of such a point to the origin is \( \sqrt{s_i^2 +
s_j^2} \).  By \eqref{eq:wallisformula:oddeveninequality}
\[
    \sqrt{ \frac{2(i+j)}{o_n} } = \sqrt{ \frac{2(n+1)}{o_n} }.
\] bounds this distance from above.  Similarly the inner corners of \( P_n
\) are the points \( ( s_i,s_j) \) for which \( i + j = n \) and \( 0
\le i,j \le n \).  The distance of such a point to the origin is bounded
from below by
\[
    \sqrt{ \frac{2(i+j-1)}{e_n} } = \sqrt{ \frac{2(n-1)}{e_n} }.
\] Therefore, \( P_n \) contains a quarter circle of radius \( \sqrt{2(n-1)/e_n}
\) and is contained in a quarter circle of radius \( \sqrt{2(n+1)/o_n} \).
See Figure~%
\ref{fig:wallisformula:proofdiagram} for a diagram of the polygonal
region \( P_n \) and the corresponding inner and outer quarter circles
for \( n = 4 \).

\begin{figure}
    \centering
    \begin{asy}
import graph;

size(5inches);

real myfontsize = 10;
real mylineskip = 1.2*myfontsize;
pen mypen = fontsize(myfontsize, mylineskip);
defaultpen(mypen);

xaxis("$x$", xmin=0, xmax=3, arrow=Arrow());
yaxis("$y$", ymin=0, ymax=3, arrow=Arrow());

real s1 = 1;
real s2 = 3/2;
real s3 = 15/8;
real s4 = 105/48;

xequals(Label("$s_1$",(0,s1)), s1, ymax=3, dotted);
yequals(Label("$s_1$",(0,s1)), s1, xmax=3, dotted);
xequals(Label("$s_2$",(0,s2)), s2, ymax=3, dotted);
yequals(Label("$s_2$",(0,s2)), s2, xmax=3, dotted);
xequals(Label("$s_3$",(0,s3)), s3, ymax=3, dotted);
yequals(Label("$s_3$",(0,s3)), s3, xmax=3, dotted);
xequals(Label("$s_4$",(0,s4)), s4, ymax=3, dotted);
yequals(Label("$s_4$",(0,s4)), s4, xmax=3, dotted);

draw( (0,s4)--(s1,s4)--(s1,s3)--(s2,s3)-- \
(s2,s2)--(s3,s2)--(s3,s1)--(s4,s1)--(s4,0) );

real Router = sqrt( (2*(4+1))/( (2^2*4^2*6^2*8)/(1*3^2*5^2*7^2) ) );
label("$r=\sqrt{2(4+1)/o_4}$", (Router,0),SE);
real Rinner = sqrt( (2*(4-1))/( (2^2*4^2*6^2)/(1*3^2*5^2*7) ) );
label("$r=\sqrt{2(4-1)/e_4}$", (Rinner,0),S);

draw( arc( (0,0), Rinner, 0, 90));
draw( arc( (0,0), Router, 0, 90));

label("$R_{0,0}$", (s1/2, s1/2));
label("$R_{0,1}$", (s1/2, (s1+s2)/2));
label("$R_{0,2}$", (s1/2, (s2+s3)/2));
label("$R_{0,3}$", (s1/2, (s3+s4)/2));

label("$R_{1,0}$", ((s1+s2)/2, s1/2));
label("$R_{1,1}$", ((s1+s2)/2, (s1+s2)/2));
label("$R_{1,2}$", ((s1+s2)/2, (s2+s3)/2));

label("$R_{2,0}$", ((s2+s3)/2, s1/2));
label("$R_{2,1}$", ((s2+s3)/2, (s1+s2)/2));

label("$R_{3,0}$", ((s3+s4)/2, s1/2));


    \end{asy}
    \caption{A diagram of the regions $ R_{i,j} $ and the inner and
    outer quarter circles for the case $ n=4 $.  %
    \label{fig:wallisformula:proofdiagram}}
\end{figure}

The area of a quarter circle of radius \( r \) is \( \PI r^2 \) and the
area of \( P_n \) is \( n \).  This leads to the bounds
\begin{equation}
    \frac{(n-1)\PI}{2e_n} < n < \frac{(n+1)\PI}{2o_n}%
    \label{eq:wallisformula:circlebounds}
\end{equation}
from which it follows that
\[
    \frac{(n-1)\PI}{2n} < e_n < o_n < \frac{(n+1)\PI}{2n}.
\] Then as \( n \to \infty \), \( e_n \) and \( o_n \) both approach \(
\PI/2 \).

\subsection*{Wallis' Formula and the Central Binomial Coefficient}

This subsection gives a detailed proof that Wallis' Formula gives an
explicit inequality bound on the central binomial term that in turn
implies the asymptotic formula for the central binomial coefficient. The
derivation in
\cite{hirschhorn15}.%
\index{central binomial coefficient}
motivates this proof.

Start from the expansion \eqref{eq:wallisformula:centralbinomexpansion}
for the Central Binomial Coefficient:
\[
    w_n = \prod_{j=1}^{n} \frac{(2j)^2}{(2j-1)(2j+1)} = \frac{ 2^{4n}}{%
    {\binom{2n}{n}}^2(2n+1)}.
\] Rearrange it to
\[
    {\binom{2n}{n}}^2 = \frac{16^n}{2n+1} \prod_{j=1}^{n} \frac{(2j-1)(2j+1)}
    {(2j)^2}
\] and use the definitions \eqref{eq:wallisformula:defnodds} and \eqref{eq:wallisformula:defnevens}
of the partial products of the Wallis formula to obtain
\[
    {\binom{2n}{n}}^2 = \frac{16^n}{(2n+1)} \frac{2n+2}{(2n+1)o_{n+1}}
\] and
\[
    {\binom{2n}{n}}^2 = \frac{16^n}{(2n+1)} \frac{1}{e_{n+1}}.
\] Rearrange the inequality \eqref{eq:wallisformula:circlebounds} to
obtain
\[
    \frac{2n+2}{\PI(n+2)} < \frac{1}{o_{n+1}} \text{ and } \frac{1}{e_{n+1}}
    < \frac{2n+2}{n\PI}.
\] Then
\[
    \frac{16^n}{(2n+1)} \frac{(2n+2)^2}{(2n+1)(n+2)\PI} < {\binom{2n}{n}}^2
    < \frac{16^n}{(2n+1)} \frac{(2n+2)}{n\PI}
\] and taking square roots and slightly rearranging again
\[
    \frac{4^n}{\sqrt{(2n+1)(\PI/2)}} \sqrt{ \frac{2(n+1)^2}{(2n+1)(n+2)}}
    < {\binom{2n}{n}} < \frac{4^n}{\sqrt{(2n+1)(\PI/2)}} \sqrt{\frac{(n+1)}
    {n}}.
\] To simplify, take a series expansion of the square roots of the
rational expressions and truncate, leaving
\[
    \frac{4^n}{\sqrt{(2n+1)(\PI/2)}}\left(1-\frac{1}{2n} \right) < {\binom
    {2n}{n}} < \frac{4^n}{\sqrt{(2n+1)(\PI/2)}} \left(1 + \frac{1}{2n}
    \right).
\]

\subsection*{A proof using the product expansion of the sine function}

\begin{theorem}
    The expansion of \( \sin(z) \) as an infinite product is
    \[
        \sin( z ) = z \prod_{m=1}^{\infty} \left( 1 - \frac{z^2}{m^2 \PI^2}
        \right).
    \]
\end{theorem}
\index{sine product expansion}

\begin{proof}
    See
    \cite[page 312]{saks71} for the proof.  The article in the
    Mathworld.com article on \link{Sine}{http://mathworld.wolfram.com/Sine.html}
    also gives references to Edwards 2001, pages 18 and 47; and Borwein
    et al.  2004, page 5.
\end{proof}

Although the common proof uses complex analysis, as in the texts cited
above, a proof using only elementary analysis is possible.  The
following proof is adapted from
\cite{kortram96}.

\begin{proof}
    Start with the definition of the Chebyshev polynomials \( T_n(x) \)
    from the trigonometric identity \( \cos(nx) = T_n( \cos(x) ) \).
    Then
    \[
        \cos( 2kx ) = T_k(\cos(2x)) = T_k(1 - 2 \sin^2 x).
    \] Together with the sine product identity
    \[
        \sin( (2k+1)x) - \sin( (2k-1)x) = 2 \sin(x) \cdot \cos(2kx)
    \] this inductively shows that there is a polynomial \( F_m \) of
    degree \( m \) such that
    \[
        \sin( (2m+1)x ) = \sin(x) \cdot F_m(\sin^2(x)).
    \] (In fact, using the definition \( U_n(\cos(x)) = \frac{\sin((n+1)x)}
    {\sin(x)} \) for the Chebyshev polynomials of the second kind, it is
    easy to show that \( F_m(x) = U_{2n}(\sqrt{1-x}) \).) Substituting \(
    x_k = \frac{k\PI}{2m+1} \) and noting that \( \sin((2m+1)x_k) = 0 \)
    shows that \( F_m \) has zeros at \( \sin^2\left((\frac{k\PI}{2m+1}
    \right) \) for \( k = 1,2,\dots,m \).  These zeros are distinct, so \(
    F_m \) has no other zeros, then
    \[
        F_m(y) = F_m(0) \prod_{k=1}^{m} \left( 1 - \frac{y}{\sin^2\left(\frac
        {k\PI}{2m+1}\right)} \right)
    \] and
    \[
        F_m(0) = \lim_{x \to 0 } \frac{\sin( (2m+1)x)}{\sin(x)} = 2m+1.
    \] Therefore
    \[
        \sin( (2m+1)x) = (2m+1) \sin(x) \prod_{k=1}^{m} \left( 1 - \frac
        {\sin^2 (x)}{\sin^2\left(\frac{k\PI}{2m+1}\right)} \right)
    \] and changing variables
    \begin{equation}
        \label{eq:wallisformula:sineprod} \sin(x) = (2m+1) \sin\left(
        \frac{x}{2m+1} \right) \prod_{k=1}^{m} \left( 1 - \frac{\sin^2\left
        (\frac{x}{2m+1}\right)}{\sin^2\left( \frac{k\PI}{2m+1}\right)}
        \right).
    \end{equation}

    The goal now is to estimate the product terms.  For all real \( t \),
    \( \EulerE^t \ge 1 + t \) and therefore \( \EulerE^{-t} \le \frac{1}
    {1 + t} \).  For \( u < 1 \), the choice \( t = u/(1-u) \) gives \(
    \EulerE^{-u/(1-u)} \le 1 - u \).  Then for every collection of
    numbers \( u_k \in [0,1) \), we have
    \begin{equation}
        \label{eq:wallisformula:expineqone} 1 - \sum_{k} \frac{u_k}{1-u_k}
        \le \EulerE^{-\sum_{k} \frac{u_k}{1-u_k}} \le \prod_{k} (1 - u_k)
        \le \EulerE^{-\sum_{k} u_k} \le 1.
    \end{equation}
    If also \( \sum_k u_k < 1 \), then we also know that
    \begin{equation}
        \label{eq:wallisformula:expineqtwo} \EulerE^{-\sum_{k} u_k} \le
        \frac{1}{1 + \sum_{k} u_k}
    \end{equation}
    and subtracting the first and third terms of \eqref{eq:wallisformula:expineqone}
    from \( 1 \) and using \eqref{eq:wallisformula:expineqtwo}
    \begin{equation}
        \label{eq:wallisfunction:sumprodineq} \frac{\sum_{k} u_k}{1 +
        \sum_{k} u_k} \le 1 - \prod_{k} (1 - u_k) \le \sum_{k} \frac{u_k}
        {1-u_k}.
    \end{equation}

    Let \( m \) and \( N \) be positive integers with \( m > N \).  Take
    \( x \) such that \( |x| < \frac{1}{4} N \PI \) and \( \frac{x}{\PI}
    \notin \Integers \).  Then define \( u_k \) by
    \[
        u_k = \left( \frac{\sin\left( \frac{x}{2m+1} \right)}{\sin\left(
        \frac{k\PI}{2m+1} \right)} \right)^2, \qquad k = 1,2,\dots,m.
    \] Use \eqref{eq:wallisformula:sineprod} by dividing the leading
    factor and the first \( N \) factors onto the left side to obtain
    \[
        \frac{\sin(x)}{(2m+1) \sin\left( \frac{x}{2m+1} \right) \prod_{k=1}^
        {N}(1-u_k)} = \prod_{k=N+1}^{m} (1-u_k).
    \] Then use the first, third and fifth terms of \eqref{eq:wallisformula:expineqone}
    to see that
    \begin{equation}
        \label{eq:wallisformula:ineqfour} 1 - \sum_{k=N+1}^{m} \frac{u_k}
        {1-u_k} \le \frac{\sin(x)}{(2m+1) \sin\left( \frac{x}{2m+1}
        \right) \prod_{k=1}^{N}(1-u_k)} \le 1.
    \end{equation}
    For \( 0 \le t \le \frac{\PI}{2} \) we have \( \sin(t) \ge \frac{2}{\PI}
    t \), so we see that
    \[
        u_k \le \left( \frac{(2m+1)\sin\left( \frac{x}{2m+1} \right)}{2k}\right)^2
        \le \left( \frac{x}{2k} \right)^2;
    \] thus
    \[
        \frac{u_k}{1-u_k} \le \frac{x^2}{(2k)^2 - x^2} \qquad \text{for}
        k > N.
    \] Hence
    \[
        \sum_{k=N+1}^{m} \frac{u_k}{1-u_k} \le \frac{x^2}{2N-|x|}.
    \] Thus it follows from \eqref{eq:wallisformula:ineqfour} that
    \[
        1 - \frac{x^2}{2N-|x|} \le \frac{\sin(x)}{(2m+1) \sin\left(
        \frac {x}{2m+1} \right) \prod_{k=1}^{N}(1-u_k)} \le 1.
    \] Let \( m \to \infty \) so that
    \[
        1 - \frac{x^2}{2N-|x|} \le \frac{\sin(x)}{x \prod_{k=1}^{N}(1-u_k)}
        \le 1.
    \] Now let \( N \to \infty \) and we obtain for \( \frac{x}{\PI}
    \notin \Integers \)
    \[
        \sin x = x \prod_{k=1}^{\infty} \left( 1 - \frac{x^2}{k^2 \PI^2}
        \right).
    \] For \( \frac{x}{\PI} \in \Integers \) this equality is also true.
\end{proof}

The following somewhat probabilistic proof of Euler's infinite product
formula is adapted from
\cite{holst12}.
\begin{proof}
    Start with the integral
    \[
        \int_{0}^{n} x^{t-1}\left( 1 - \frac{x}{n} \right)^n \df{x}
    \] and integrate by parts once:
    \begin{align*}
        u &= \left(1-\frac{x}{n}\right)^n & v &= \frac{1}{t} x^{t} \\
        \df{u} &= -\left(1-\frac{x}{n}\right)^{n-1} \df{x} & \df{v} &= x^
        {t-1} \df{x}.  \\
    \end{align*}
    Then
    \begin{align*}
        \int_{0}^{n} x^{t-1}\left( 1 - \frac{x}{n} \right)^n \df{x} &=
        \left.  \left(1-\frac{x}{n}\right)^n \right|_{0}^{n} + \int_{0}^
        {n} \frac{1}{t} x^{t} \left(1-\frac{x}{n}\right)^{n-1} \df{x} \\
        &= \frac{1}{t} \int_{0}^{n} x^{t} \left(1-\frac{x}{n}\right)^{n-1}
        \df{x}
    \end{align*}
    Integrate by parts again:
    \begin{align*}
        u &= \left(1-\frac{x}{n}\right)^{n-1} & v &= \frac{1}{t+1} x^{t+1}
        \\
        \df{u} &= -\frac{n-1}{n} \left(1-\frac{x}{n}\right)^{n-2} \df{x}
        & \df{v} &= x^{t} \df{x} \\
    \end{align*}
    so
    \[
        \int_{0}^{n} x^{t-1}\left( 1 - \frac{x}{n} \right)^n \df{x} =
        \frac{(n-1)}{t(t+1)n} \int_{0}^{n} x^{t+1} \left(1-\frac{x}{n}\right)^
        {n-2} \df{x}.
    \] After integration by parts \( n \) times:
    \begin{align*}
        \int_{0}^{n} x^{t-1}\left( 1 - \frac{x}{n} \right)^n \df{x} &=
        \frac{(n-1)!}{t(t+1) \dots (t+n-1) n^{n-1}} \int_0^n x^{t+n-1}
        \df{x} \\
        &= \frac{(n-1)!}{t(t+1) \dots (t+n-1) n^{n-1}} \left.  \frac{x^{t+n}}
        {t+n} \right|_0^n \\
        &= \frac{n!  \, n^t}{t(t+1) \dots (t+n)}.
    \end{align*}

    Then by dominated convergence it follows that
    \[
        \Gamma(t) = \int_0^{\infty} x^{t-1} \EulerE^{-x} \df{x} = \lim_{n
        \to \infty} \frac{ n!\,n^t }{t(t+1)\dots(t+n)}.
    \] Note that limit definition of the Gamma function is also \link{http://dlmf.nist.gov/5.8
    E1}{http://dlmf.nist.gov/5.8 E1}.  It follows that for \( -1 < t < 1
    \) that
    \[
        \Gamma(1 + t) \Gamma(1-t) = \prod_{k=1}^{\infty} \frac{1}{1 - t^2/k^2}.
    \]

    For \( X \) and \( Y \) independent exponential random variables,
    both with expectation \( 1 \) we have that
    \[
        \E{X^t} = \int_0^\infty x^t \EulerE^{-x} \df{x} = \Gamma(1 + t)
    \] and likewise \( \E{Y^{-t}} = \Gamma(1-t) \).  Then using the
    independence and symmetry
    \[
        \Gamma(1+t)\Gamma(1-t) = \E{X^t}\E{Y^{-t}} = \E{ (X/Y)^t } = \E{%
        (X/Y)^{|t|} }
    \]

    The distribution function for the random variable \( X/Y \) is
    \[
        \Prob{ X/Y \le u } = \Prob{ Y \ge X/u } = \E{ \EulerE^{-X/u}} =
        \frac{u}{u+1}
    \] for \( u > 0 \).  Then the probability density is
    \[
        \deriv{}{u} \Prob{X/Y \le u} = \frac{1}{(1+u)^2}, \qquad u > 0.
    \] This gives by integration by parts, for \( -1 < t < 1 \), that
    \[
        \E{(X/Y)^{|t|}} = \int_{0}^{\infty} \frac{u^{|t|}}{(u+1)^2} \df{u}
        = \int_{0}^{\infty} \frac{ |t| u^{|t|-1}}{u+1} \df{u}
    \]

    The last integral is \( \PI t/\sin(\PI t) \).  For example, this is
    formula 613, page 445 in the 18th Edition of the CRC Standard
    Mathematical Tables.  It can also be done with contour integration
    in the complex plane
    \[
        \int_{0}^{+\infty} \frac{ |t| u^{|t|-1}}{u+1} \df{u} - 2\PI \I (-1)^
        {|t|-1} + \int_{+\infty}^0 \frac{ |t| u^{2\PI\I(|t|-1)}}{u+1}
        \df{u} = 0
    \] so
    \[
        \int_{0}^{\infty} \frac{ |t| u^{|t|-1}}{u+1} \df{u} \left( 1-
        \EulerE^{2\PI \I |t|} \right) = -2 \PI \I \EulerE^{\PI \I |t|}.
    \] Thus
    \[
        \int_{0}^{\infty} \frac{ |t| u^{|t|-1}}{u+1} \df{u} = \frac{|t|
        2 \PI \I \EulerE^{\PI \I |t|}}{\EulerE^{2\PI \I |t|} - 1} =
        \frac{ \PI |t| }{\sin ( \PI |t| )} = \frac{ \PI t}{\sin( \PI t)}.
    \] Hence, for \( -1 < t < 1 \)
    \[
        \Gamma(1 + t) \Gamma(1 - t) = \E{ (X/Y)^t} = \frac{ \PI t}{\sin(
        \PI t)}.
    \] A standard integration-by-parts gives the fundamental Gamma
    function recursion \( \Gamma(t) = \Gamma(1+t)/t \).  Using this to
    define \( \Gamma(t) \) for \( t < 0 \), it follows that for all real
    \( t \),
    \[
        \sin( \PI t ) = \frac{\PI}{\Gamma(t)\Gamma(1-t)}.
    \] Combining all results above,
    \[
        \sin( \PI t ) = \PI t \prod_{k=1}^{\infty} \left( 1 - \frac{t^2}
        {k^2} \right).
    \]
\end{proof}

\begin{corollary}[Wallis' Formula]
    \[
        \lim_{n \to \infty} \frac{ 2 \cdot 2 \cdot 4 \cdot 4 \cdot 6
        \cdot 6 \dots (2n) \cdot (2n)}{1 \cdot 3 \cdot 3 \cdot 5 \cdot 5
        \dots (2n-1) \cdot (2n-1)\cdot(2n+1)} = \frac{\PI}{2}.
    \]
\end{corollary}
\index{Wallis's Formula}

\begin{proof}
    Substitute \( z = \PI/2 \) in the product expansion of \( \sin(z) \).
\end{proof}

\begin{remark}
    In
    \cite{ciaurri15} Ciaurri uses Tannery's Theorem for Infinite
    Products, a trig identity for cotangent and tangent, and Wallis
    formula to provide an elementary proof of product expansion of the
    sine function.  In this sense, the product expansion of the sine
    function is equivalent to Wallis' formula.
\end{remark}

\subsection*{A Geometric Interpretation of Wallis' Formula}

This section gives a geometric interpretation of Wallis' formula,
constructing a subset of the unit square with area \( \frac{8}{9} \cdot
\frac{24}{25} \cdot \frac{48}{49} \cdots = \frac{2}{3} \cdot \frac{4}{3}
\cdot \frac{4}{5} \cdot \frac{6}{5} \cdot \frac{6}{7} \cdot \frac{8}{7}
\cdots = \frac{1}{2} \cdot \frac{\pi}{2} \).  The construction is
reminiscent of the construction of the Sierpinski Triangle.

Take a unit square \( W_0 \) and remove a square of area \( \frac{1}{3}
\cdot \frac{1}{3} \) from the center.  Label the result as \( W_1 \)
with area \( \frac{8}{9} \).  Next remove a square of area \( \frac{1}{5}
\cdot \frac{1}{5} \) from the center of each of the \( 8 \) small
sub-squares of area \( \frac{1}{9} \) constituting \( W_1 \) and call
the result \( W_2 \) with area \( \frac{8}{9} \cdot \frac{24}{25} \).
Continuing by removing squares of area \( \frac{1}{7} \cdot \frac{1}{7} \)
from each sub-square of area \( \frac{1}{5} \cdot \frac{1}{5} \)
surrounding the hole removed at stage \( 2 \), construct region \( W_3 \)
with area \( \frac{8}{9} \cdot \frac{24}{25} \cdot \frac{48}{49} \).
Continue to create \( W_4 \), and so on.  Call the limiting region \( W_
{\infty} \) which has area \( \frac{8}{9} \cdot \frac{24}{25} \cdot
\frac{48}{49} \cdots = \frac{1}{2} \cdot \frac{\pi}{2} \).

%% The number of squares removed at stage $n$ is $2^{2(n-1)}n!/(n-1)!$.
The side length of the removed squares at stage \( n \) is \( \frac{2^n
n!}{(2n+1)!} \).  After \( 3 \) stages, \( 201 \) squares have been
removed, and the stage \( 3 \) holes have side length \( \frac{1}{3}
\cdot \frac{1}{5} \cdot \frac{1}{7} = \frac{1}{105} \).  It will be
impractical to illustrate more than \( 3 \) stages of this iterative
process.

\begin{figure}[htbp]
\begin{asy}
size(5inches);

real myfontsize = 12;
real mylineskip = 1.2*myfontsize;
pen mypen = fontsize(myfontsize, mylineskip);
defaultpen(mypen);

path us = unitsquare;
int NStages = 3;

real [] factor;
transform [] scalefactor;
transform [][] offset = new transform[4][];
path [] stage;

factor[0] = 1;

for(int i=1; i <= NStages; ++i) {
  factor[i] = factor[i-1] * (1/(2*i+1)); 
  scalefactor[i] = scale(factor[i]);
  for(int k=0; k<2*i+1; ++k) {
    for(int j=0; j<2*i+1; ++j) {
      offset[i][(2*i+1)*k + j] = shift(j*factor[i], k*factor[i]);
    }
  }
}

stage.push( us );
stage.push( offset[1][(2*1+1)*1+1]*scalefactor[1]*us );
filldraw(stage, lightgrey+evenodd, black);

for(int k=0; k<2*1+1; ++k) {
  for(int j=0; j<2*1+1; ++j) {
    if ( j == 1 && k == 1) {
      continue;
    }
    else {
    stage.push( offset[2][(2*2+1)*2+2] * offset[1][(2*1+1)*k+j] * scalefactor[2]*us );
    }
  }
}
filldraw(shift(1.5, 0) * stage, lightgrey+evenodd, black);

for(int k=0; k<2*1+1; ++k) {
  for(int j=0; j<2*1+1; ++j) {
    if ( j == 1 && k == 1) {
      continue;
    }
    else {
      for(int m=0; m<2*2+1; ++m) {
	for(int n=0; n<2*2+1; ++n) {
	  if ( m == 2 && n == 2) {
	    continue;
	  }
	  else {
	    stage.push( offset[3][(2*3+1)*3+3] *
		  offset[2][(2*2+1)*m+n] *
		  offset[1][(2*1+1)*k+j] * scalefactor[3]*us );
	  }
	}
      }
    }
  }
}
filldraw(shift(3, 0) * stage, lightgrey+evenodd, black);
\end{asy}

    \caption[]{Three stages in the construction of the Wallis sieve.}
\end{figure}

Here area means the Lebesgue measure of the limiting region \( W_{\infty}
\). However \( W_{\infty} \) does not contain any product set \( A
\times B \) with \( A, B \subset \Reals \), each with positive Lebesgue
measure.  To see this, consider a maximal product subset of \( W_1 \),
for example \( [0,1] \times ([0,\frac{1}{3}] \union [\frac{2}{3}, 1]) \)
(or equivalently \( ([0,\frac{1}{3}] \union [\frac{2}{3}, 1]) \times [0,1]
\) although from here on, consider only the product sets with first
factor \( [0,1] \)). This product subset has measure \( \frac{2}{3} \).
A similar maximal product subset of \( W_2 \) has measure \( \frac{2}{3}
\cdot \frac{4}{5} \).  Continuing, a maximal product subset of \( W_{\infty}
\) has measure \( \frac{2}{3} \cdot \frac{4}{5} \cdot \frac{6}{7} \cdots
= \prod\limits_{n=1}^{\infty} \left( 1- \frac{1}{2n+1}\right) = 0 \).

Thus, \( W_{\infty} \) is an example of a subset of the plane \( \Reals^2
\) with positive Lebesgue measure, but not admitting any product subset
with positive Lebesgue measure.  Note also that \( W_{\infty} \) is a
``square-like'' region with area equal to a circle with radius \( \frac{1}
{2} \).  In that sense, this is a solution to the ancient problem of
``squaring the circle''.

\subsection*{Sources}

This section is adapted from a sketch of the proof of Wallis' Formula in
Kazarinoff
\cite{kazarinoff61} and the short note by W\"{a}stlund
\cite{wastlund07}.  The elementary proofs of the sine product are from
\cite{kortram96} and
\cite{holst12}. The geometric interpretation of Wallis' Formula is
adapted from
\cite{rummler}.

\hr

\visual{Problems to Work}{../../../../CommonInformation/Lessons/solveproblems.png}
\section*{Problems to Work for Understanding}
\begin{enumerate}
    \item
        Show that
        \[
            \frac{ 2^{4n} \cdot (n!)^4}{((2n)!)^2(2n+1)} = \frac{ 2
            \cdot 2 \cdot 4 \cdot 4 \cdot 6 \cdot 6 \dots (2n) \cdot (2n)}
            {1 \cdot 3 \cdot 3 \cdot 5 \cdot 5 \dots (2n-1)\cdot(2n+1)}.
        \]%
    \item
        Numerically find the rate at which the sequence
        \[
            \frac{ 2 \cdot 2 \cdot 4 \cdot 4 \cdot 6 \cdot 6 \dots (2n)
            \cdot (2n)}{1 \cdot 3 \cdot 3 \cdot 5 \cdot 5 \dots (2n-1)\cdot
            (2n+1)}
        \] approaches \( \PI/2 \) by calculating values and using
        regression.
    \item
        Use induction to prove
        \[
            J_{2n+1} = \frac{(2n)\cdot (2n-2) \dots 4 \cdot 2}{(2n+1)
            \cdot (2n-1) \dots 3 \cdot 1}
        \] and
        \[
            J_{2n} = \frac{(2n-1)\cdot(2n-3) \dots 3 \cdot \PI}{(2n)\cdot
            (2n-2)\dots 4 \cdot 2 \cdot 2}.
        \]
    \item
        The simple proof of the sine product formula uses the polynomial
        \( F_m \) of degree \( m \) such that
        \[
            \sin( (2m+1)x ) = \sin(x) \cdot F_m(\sin^2(x)).
        \] Explicitly find \( F_0(x), F_1(x), F_2(x), F_3(x), F_4(x) \)
        and plot them on \( [-1,1] \).
        % Solution:  F_0(x) = 1, F_1(x) = 3 - 4x, F_2(x) = 16x^2 -
        % 20x^2 + 5, F_3(x) = -64x^3 + 112 x^2 - 56 x + 7, F_4(x)
        % 256x^4 - 576 x^3 + 432 x^2 -120 x + 9
    \item
        Give a detailed proof that
        \[
            \int_{0}^{n} x^{t-1}\left( 1 - \frac{x}{n} \right)^n \df{x}
            = \frac{n!  \, n^t}{t(t+1) \dots (t+n)}.
        \]
    \item
        Provide a careful and detailed proof that if \( X \) and \( Y \)
        are independent exponential random variables, both with
        expectation \( 1 \) we have that
        \[
            \E{ (X/Y)^t } = \E{(X/Y)^{|t|} }.
        \]
    \item
        Provide a detailed proof using contour integration in the
        complex plane to show that
        \[
            \int_{0}^{\infty} \frac{ |t| u^{|t|-1}}{u+1} \df{u} = \frac{|t|
            2 \PI \I \EulerE^{ \PI \I\ |t|}}{\EulerE^{2\PI \I |t|} - 1}
            = \frac{ \PI t}{\sin ( \PI t)}.
        \]

\end{enumerate}

%% \link{ _soln.xml}{Solutions to Problems}

\hr

\visual{Books}{../../../../CommonInformation/Lessons/books.png}
\section*{Reading Suggestion:}

\bibliography{../../../../CommonInformation/bibliography}

%   \begin{enumerate}
%     \item
%     \item
%     \item
%   \end{enumerate}

\hr

\visual{Links}{../../../../CommonInformation/Lessons/chainlink.png}
\section*{Outside Readings and Links:}
\begin{enumerate}
    \item
    \item
    \item
    \item
\end{enumerate}

\hr

\mydisclaim \myfooter

Last modified:  \flastmod

\end{document}

readability grades:
        Kincaid: 10.5
        ARI: 11.0
        Coleman-Liau: 13.2
        Flesch Index: 52.3
        Fog Index: 13.5
        Lix: 46.7 = school year 8
        SMOG-Grading: 12.2
sentence info:
        8904 characters
        1809 words, average length 4.92 characters = 1.60 syllables
        98 sentences, average length 18.5 words
        51% (50) short sentences (at most 13 words)
        17% (17) long sentences (at least 28 words)
        37 paragraphs, average length 2.6 sentences
        2% (2) questions
        44% (44) passive sentences
        longest sent 81 wds at sent 45; shortest sent 1 wds at sent 22
word usage:
        verb types:
        to be (56) auxiliary (3) 
        types as % of total:
        conjunctions 5% (87) pronouns 4% (80) prepositions 12% (212)
        nominalizations 4% (72)
sentence beginnings:
        pronoun (7) interrogative pronoun (1) article (17)
        subordinating conjunction (5) conjunction (1) preposition (8)

File name                  : wallisformula.tex
Number of characters       : 32927
Number of words            : 3260
Percent of complex words   : 14.97
Average syllables per word : 1.6626
Number of sentences        : 65
Average words per sentence : 50.1538
Number of text lines       : 698
Number of blank lines      : 197
Number of paragraphs       : 189


READABILITY INDICES

Fog                        : 26.0493
Flesch                     : 15.2749
Flesch-Kincaid             : 23.5884


